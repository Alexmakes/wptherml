\documentclass[journal=jpclcd,manuscript=suppinfo]{achemso}
\pdfoutput=1
\usepackage{gensymb}
\usepackage{amsmath}
\usepackage{graphicx}
\usepackage{epsfig}
\usepackage{color}
%%\renewcommand{\thefigure}{S\arabic{figure}}
\author{James F. Varner, Noor Eldabagh, Derek Volta, Reem Eldabagh, Jonathan J. Foley IV}
\affiliation{Department of Chemistry, William Paterson University, 300 Pompton Road, Wayne, NJ, 07470, USA}
\email{foleyj10@wpunj.edu}
\title{Mathematical Basis of wptherml}
\date{\today}

\begin{document}
\section{Transfer Matrix Equations}
The Transfer Matrix Equations are solved to obtain optical quantities
of a multi-layer stack, including the reflectivity, transmissivity,
and emissivity.  The Transfer Matrix equatons can be written as~\cite{Yeh}:
\begin{equation}
  \begin{pmatrix}
  E_1^+ \\
  E_1^-
  \end{pmatrix} 
  \mbox{=}
  \begin{pmatrix}
    M_{11}    &     M_{12}   \\
    M_{21}    &     M_{22}   
  \end{pmatrix}
  \begin{pmatrix}
  E_L^+ \\
  E_L^-,
  \end{pmatrix}
\end{equation}
where $E_1^+$ and $E_1^-$ are the incident and reflected field amplitudes,
respectively;  $E_L^+$ is the transmitted field amplitude and $E_L^-$ is set
to zero.  The Transfer Matrix is a product of matrices for each layer
in the multi-layer stack:
\begin{equation}
\begin{pmatrix}
    M_{11}    &     M_{12}   \\
    M_{21}    &     M_{22}   
  \end{pmatrix}
\mbox{=}
{\bf D}_1^{-1} \left( \prod_{l=2}^{L-1} {\bf D}_l {\bf P}_l {\bf D}_l^{-1}
\right) {\bf D}_L.
\end{equation}
The terminal layers, Layer 1 and Layer $L$, are taken to be semi-infinite, and the central
layers (Layer 2 - $L-1$) have finite thickness.  Furthermore,
the terminal layers have real refractive indices, while the central
layers may have complex refractive indices.
The ${\bf P}$ matrix is defined for each of the central
layers:
\begin{equation}
{\bf P}_l = 
\begin{pmatrix}
    {\rm exp}(-i k_{z,l} d_l)    &     0   \\
    0    &     {\rm exp}(i k_{z,l} d_l)    
\end{pmatrix}
\end{equation}
where 
\begin{equation}
k_{z,l} = \sqrt{\left(n_l \frac{\omega}{c}  \right)^2 - \left( n_l {\rm sin}(\theta_1) \frac{\omega}{c} \right)^2 },
\end{equation}
$d_l$ and $n_l$ is the thickness and refractive index of layer $l$, respectively, 
$\omega$ is the frequency of light, $\theta_1$ is the angle of incidence upon the 
multi-layer structure, and $c$ is the vacuum speed of light.  We note that 
$n_l$ often depends upon $\omega$ for materials of interest.  
The ${\bf D}$ matrices are defined for all layers, and depend upon the incident
polarization of light.  For p-polarized light:
\begin{equation}
{\bf D}_l = 
\begin{pmatrix}
    {\rm cos}(\theta_l)    &     {\rm cos}(\theta_l)   \\
    n_l    &     -n_l   
\end{pmatrix}
\end{equation}
and for s-polarized light:
\begin{equation}
{\bf D}_l = 
\begin{pmatrix}
    1 &     1   \\
    n_l {\rm cos}(\theta_l)    &     -n_l {\rm cos}(\theta_l)    
\end{pmatrix}
\end{equation}
where $\theta_l$ is the refraction angle into the $l^{th}$ layer that satisfies Snell's 
la: $\theta_l = {\rm arcsin}(n_1/n_l \: {\rm sin}(\theta_1))$.

From the elements of the Transfer Matrix, the reflection and transmission amplitudes
can be computed as follows:
\begin{align}
r = \frac{M_{21}}{M_{11}} \\
t = \frac{1}{M_{11}}.
\end{align}
The far-field reflectivity is computed from the reflection amplitude as
\begin{equation}
R = r r^*
\end{equation}
and the far-field transmissivity is computed from the transmission amplitude and 
incident/refraction angles as 
\begin{equation}
T = t t^* \: \frac{n_L \: {\rm cos}(\theta_L)}{n_1 \: {\rm cos}(\theta_1)}
\end{equation}
where $r^*$ and $t^*$ are the complex conjugates of $r$ and $t$, respectively.
The Absorptivity and Emissivity of a structure can be computed as
\begin{equation}
A \equiv \epsilon = 1 - R - T,
\end{equation}
where $A$ indicates the Absorptivity and $\epsilon$ is the emissivity, which are taken
to be equivalent by Kirchoff's theorem.

The Thermal Emission of a given structure is simply the emissivity multiplied
by the Planck's blackbody function.  We note from the above that the emissivity
can be seen to depend upon frequency/wavelength and angle; it also depends
upon temperature through the blackbody function.  Most generally, the thermal emission
spectrum as a function of wavelength, angle, and temperature is
\begin{equation}
TE(\lambda, \theta, T) = 
\frac{1}{2}\left( \epsilon_p(\lambda, \theta) + \epsilon_s(\lambda, \theta) \right) \:
\rho(\lambda, T) \: {\rm cos}(\theta)
\end{equation} 
where 
\begin{equation}
\rho(\lambda, T) = \frac{2hc^2}{\lambda^5} 
\frac{1}{{\rm exp}(\frac{hc}{\lambda k_B T})-1}
\end{equation}
is Planck's blackbody spectrum, ${\rm cos}(\theta)$ accounts
for the angular dependence of the emitted power of an ideal blackbody, and $\epsilon_p(\lambda, \theta)$ and $\epsilon_s(\lambda, \theta)$ are the emissivities for p- and s-polarizations, respectively.

In the code, we often make distinctions between quantities which are derived
from thermal emission with explicit angular dependence of the emissivity and quantities
that assume that the emissivity does not change with angle.  Strictly speaking, 
the emissivity is a function of angle so neglecting the angular dependence is an 
approximation, but one may choose to make this approximation to reduce the computational
cost associated with computing some quantities.  As an example, consider the total 
power emitted into a hemisphere by a surface.  If the emissivity depends upon angle, the
power radiated into the hemisphere per unit area of the surface is given by
\begin{equation}
P = \int_0^{2\pi} d\phi \int_0^{\pi/2} {\rm sin}(\theta) \: d\theta 
\int_{\lambda_{min}}^{\lambda_{max}} TE(\lambda, \theta, T) d\lambda.
\end{equation}   
If the emissivity does not depend on angle, one can analytically integrate out the 
angular components:
\begin{align}
P &= \int_0^{2\pi} d\phi \int_0^{\pi/2} {\rm sin}(\theta) \: {\rm cos}(\theta) \: d\theta  
\int_{\lambda_{min}}^{\lambda_{max}} \epsilon(\lambda) \rho(\lambda, T) d\lambda \\
 &= \pi \int_{\lambda_{min}}^{\lambda_{max}} \epsilon(\lambda) \rho(\lambda, T) d\lambda.
\end{align}
Calculation of the emissivity spectrum, while relatively efficient,  
constitutes the main computational bottleneck in wptherml, so avoiding
the need to compute the spectrum for a range of different angles and polarizations
presents an advantage.  In the case that the angular dependence is explicitly considered,
we choose a 7th-order Gauss-Legendre grid between 0 and $\pi/2$ over which the emissivity
is evaluated; with this approach, the explicit angle quantities require 14 evaluations
of the emissivity functions (7 for each polariation) as opposed to a single
evaluation of the emissivity function when angular dependence is neglected.

With the Transfer Matrix Equations in hand, and their relation to the thermal
emission of a multi-layer structure established, we will now provide a brief overvie of the 
central equations used for the figures of merit for the various applications wptherml
can be used for.

\section{Thermophotovoltaics}
A number of figures of merit may be computed to characterize selective thermal emitters
for thermophotovoltaic applications.
\newline
Useful Power Density~\cite{JHR_AdvEnMat_2018}:
\begin{equation}
P = \pi \int_0^{\lambda_{bg}} \frac{\lambda}{\lambda_{bg}} TE(\lambda, T) d\lambda
\end{equation}
where $\lambda_{bg}$ is the bandgap wavelength of the target PV.
\newline
Spectral Efficiency~\cite{JHR_AdvEnMat_2018}:
\begin{equation}
\eta_s = \frac{P}{\pi \int_{\lambda_{min}}^{\lambda_{max}} TE(\lambda, T) d\lambda}
\end{equation}
where $P$ is the useful power density previously described, and the
denominator is the total emitted power density, $P_{emit}$.
\newline
TPV Efficiency~\cite{QHM_Solar_2005, RF_OptExp_2009,JHR_AdvEnMat_2018}:
\begin{equation}
\eta_{TPV} = \frac{ J_{sc} V_{oc} FF}{P_{emit}}.
\end{equation}
The emitted power density in the denminator is the same as the denominator
in the spectral efficiency;  
the short circuit current is~\cite{QHM_Solar_2005, RF_OptExp_2009,JHR_AdvEnMat_2018}
\begin{equation}
J_{sc} = \pi F \int_{\lambda_{min}}^{\lambda_{max}} TE(\lambda, T) SP(\lambda) d\lambda
\end{equation}
where $F$ is the view factor and $SP(\lambda)$ is the spectral response of the PV cell (in Amps/Watts);
the spectral response function decays to zero relatively quickly for wavengths greater than $\lambda_{bg}$, 
so the integral can be truncated. 
The Open Circuit Voltage can be approximated as~\cite{QHM_Solar_2005, JHR_AdvEnMat_2018}
\begin{equation}
V_{oc} = \frac{k_B T_{cell}}{q} {\rm Ln}\left(\frac{J_{sc}}{J_0}\right) ,
\end{equation}
where $k_B$ is Boltzmann's constant, $T_{cell}$ is the temperature of the PV cell,
and $J_0$ is the saturation current density estimated as~\cite{QHM_Solar_2005, JHR_AdvEnMat_2018}
\begin{equation}
J_0 = 1.5 \cdot 10^5 \: {\rm exp}\left( - \frac{E_{bg}}{k_B T_{cell}} \right).
\end{equation}
The Fill Factor is estimated as~\cite{QHM_Solar_2005, JHR_AdvEnMat_2018} 
\begin{equation}
FF = \beta \frac{v - {\rm Ln}(v + 0.72)}{v+1}
\end{equation}
where $v$ is the reduced $V_{oc}$~\cite{QHM_Solar_2005, JHR_AdvEnMat_2018}:
\begin{equation}
v = \frac{V_{oc}}{k_B T_{cell}}
\end{equation}
and $\beta$ is a parameter (which we take to be 0.96 following Qui {\it et al.}~\cite{QHM_Solar_2005}
\newline
Absorber Efficiency:
\begin{equation}
\eta_{abs} = \frac{P_{abs} - P_{emit}}{P_{abs}}
\end{equation}
where the total absorbed power density is given by~\cite{RF_OptExp_2009}
\begin{equation}
P_{abs} = 2\pi C \int_0^{\theta_C} d\theta \: {\rm sin}(\theta) \: {\rm cos}(\theta) 
\int_{\lambda_{min}}^{\lambda_{max}}\frac{1}{2} \left(\epsilon_p(\lambda, \theta) + \epsilon_s(\lambda, \theta) \right) AM(\lambda) d\lambda
\end{equation}
where $C$ is the solar concentration, $AM(\lambda)$ is the AM1.5 spectrum, and $\theta_C$ depends upon
the solar concentration as $\theta_C = {\rm arcsin}(C \cdot 68.5 \cdot 10^{-6} / \pi)$~\cite{RF_OptExp_2009}.

The explicit angle dependence is always included for the total absorbed power in the absorber efficiency 
calculation, and by the user's option, can be included in all other STPV figures of merit by performing
the integration over the full angle-dependent thernal emission as defined in Eq. (14).  As in the 
total absorbed power, the explicit angle dependence of the p- and s-polarized emissivities must be accounted
for in explicit integration over $\theta$; the range of $\theta$ will be from $0$ to $2\pi$ for all applications
except the total absorbed solar power, where the $\theta$ range depends upon the solar concentration as discussed above.
\section{Incandescent Lighting}
Two figures of merit are commonly used to characterize the efficiency of incandescent sources: luminous efficiency and
luminous efficacy.  Both figures of merit quantify the ratio of emitted power that is visible to the human eye to 
the total emitted power of the source.  The visible emitted power is taken to be a convolution of the photopic luminosity function
and the thermal emission spectrum.  Hence, the luminous efficiency can be defined as~\cite{IBC_NatNano_2016}:
\begin{equation}
\eta_{Lum} = \frac{ \int_{\lambda_{min}}^{\lambda_{max}} v(\lambda) TE(\lambda, T) d\lambda }{ \int_{\lambda_{min}}^{\lambda_{max}} TE(\lambda, T) d\lambda  } 
\end{equation}
where $v(\lambda)$ is the photopic luminosity function.  The Luminous efficacy is simply the luminous efficiency scaled by 683:
\begin{equation}
K = 683 \cdot \eta_{Lum}.
\end{equation} 

\section{Color Rendering}
Two methods are provided for rendering the color of a given structure: the ThermalColor method uses the thermal emission spectrum while
the AmbientColor method uses the reflection spectrum as the basis for the color rendering.  
With a given spectrum, the color can be rendered by mapping the convolution of the spectrum with the response functions of 
the three color cones of the human eye.  
For example, for thermal color, the following three convolutions are taken:
\begin{align}
X = \int_{\lambda_{min}}^{\lambda_{max}} TE(\lambda, T) C_r(\lambda) d\lambda \\
Y = \int_{\lambda_{min}}^{\lambda_{max}} TE(\lambda, T) C_g(\lambda) d\lambda \\
Z = \int_{\lambda_{min}}^{\lambda_{max}} TE(\lambda, T) C_b(\lambda) d\lambda,
\end{align} 
where $C_r(\lambda)$ is the red cone response function, etc. 
The $X, Y, Z$ values are linearly related to $r, g, b$ values which may be directly rendered as color:
\begin{equation}
  \begin{pmatrix}
  X \\
  Y \\
  Z
  \end{pmatrix} 
  \mbox{=}
  \begin{pmatrix}
    S_r X_r & S_g X_g & S_b X_b   \\
    S_r Y_r & S_g Y_g & S_b Y_b   \\
    S_r Z_r & S_g Z_g & S_b Z_b
  \end{pmatrix}
  \begin{pmatrix}
  r \\
  g \\
  b
  \end{pmatrix}.
\end{equation}
For ambient color calculations, the thermal emission spectra in Eqs. (29)-(31) are replaced with 
the reflectance spectra.
\section{Radiative Cooling}
The cooling power of the structure is approximated as~\cite{RAZ_Nature_2014}:
\begin{equation}
P_{cool}(T) = P_{rad}(T) - P_{atom}(T_{amb}) - P_{sun},
\end{equation} 
where $P_{rad}(T)$ is the power radiated by the structure at temperature $T$:
\begin{equation}
P_{rad}(T) = A \: 2\pi \int_0^{\pi/2} d\theta \: {\rm sin}(\theta) \int_{\lambda_{min}}^{\lambda_{max}} d\lambda TE(\lambda, T, \theta),
\end{equation}
where $A$ is the area of the structure, $P_{atm}(T_{amb})$ is the power absorbed by the structure that is radiated from the
atmosphere, which has a thermal emission spectrum that arises from its finite temperature ($T_{amb}$) and emissivity.  The
thermal emission from the atmosphere that is absorbed by the structure is given by
\begin{equation}
P_{atm}(T_{amb}) = A 2\pi \int_0^{\pi/2} d\theta \: {\rm sin}(\theta) {\rm cos}(\theta)\int_{\lambda_{min}}^{\lambda_{max}} 
d\lambda \left(\epsilon_p(\lambda, \theta) + \epsilon_s(\lambda, \theta)  \right) \rho(\lambda, T_{amb}) 
\epsilon_{atm}(\lambda, \theta),
\end{equation}
where $\epsilon_p(\lambda, \theta)$ is the p-polarized emissivity of the structure and $\epsilon_{atm}(\lambda, \theta)$ is the
emissivity of the atmosphere, which can be computed as 
\begin{equation}
\epsilon_{atm}(\lambda, \theta) = 1-T_{atm}(\lambda)^{1/{\rm cos}(\theta)}
\end{equation}
where $T_{atm}(\lambda)$ is the transmissivity of the atmosphere.
The solar power absorbed by the structure is given by 
\begin{equation}
P_{sun} = A \int_{\lambda_{min}}^{\lambda_{max}} d\lambda \: \frac{1}{2} \left( \epsilon_p(\lambda, \theta_{sun}) + \epsilon_s(\lambda, \theta_{sun})\right) AM(\lambda).
\end{equation}
We note that this approximation to the cooling power only accounts for radiative heating/cooling forces, and neglects
convection and conduction.

\section{Optical Resonances}
Two methods are implemented which can be used to locate surface plasmon polariton (SPP) and perfectly absorbing (PA) resonances
in multi-layer structures.  These methods utilize the fact that the p-polarized reflection amplitude tends towards inifinity 
as the SPP resonance and towards zero at the PA resonance~\cite{FHR_SciRep_2015}.  
Therefore, the resonance finders views the p-polarized reflection amplitude
as a function of the real and imaginary components of the wavevector parallel to the multi-layer surface at a particular wavelength 
and seeks either a maximum (SPP) or minimum (PA) as a function of these wavevector components.
\newline
SPP Resonance
\begin{align}
{\rm Max}\left( r (\lambda, {\rm Re}(k_x), {\rm Im}(k_x))\right) \\
{\rm s.t.} \\
n_L \frac{2\pi}{\lambda} < {\rm Re}(k_x) < n_0  \frac{2\pi}{\lambda} \; \; \; \& \; \; \; \; {\rm Im}(k_x) > 0
\end{align}
\newline
PA Resonance
\begin{align}
{\rm Min}\left( r (\lambda, {\rm Re}(k_x), {\rm Im}(k_x))\right) \\
{\rm s.t.} \\
n_L \frac{2\pi}{\lambda} < {\rm Re}(k_x) < n_0  \frac{2\pi}{\lambda} \; \; \; \& \; \; \; \; {\rm Im}(k_x) > 0
\end{align}
\bibliography{Equations}
\end{document}
